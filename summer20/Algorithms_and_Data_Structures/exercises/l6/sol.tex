\documentclass[12pt,a4paper]{article}

\usepackage[T1]{fontenc}
\usepackage[utf8]{inputenc}
\usepackage[polish]{babel}
\usepackage{indentfirst}
\usepackage{amsfonts}
\usepackage{amsmath}
\usepackage{algorithmic}
\usepackage{amssymb}

\usepackage[margin=0.5in,headheight=48pt,top=78pt]{geometry}
\frenchspacing
\setlength{\parskip}{1em}
\setlength{\parindent}{0em}
\def\N{\mathbb{N}}
\def\R{\mathbb{R}}
\newcommand{\zadanie}[1]{\par\textbf{Zadanie #1}}
\newcommand{\odp}[1]{\textbf{Odpowiedź:} #1}

\usepackage{tikz}

\usepackage{fancyhdr}
\pagestyle{fancy}
\fancyhf{} % clear all fields
\fancyhead[C]{ \textbf{AISD} - Lista 6 Zadanie 3\ Wiktor Pilarczyk 308533}

\begin{document}
\section{Wprowadzenie}
Na wykładzie poznaliśmy algorytm, który dla drzew ukorzenionych w czasie O(n) sprawdza czy nasze drzewo jest ukorzenione. Problem czy drzewa nieukorzenione sprowadzimy do tego samego problemu znajdując odpowiedni korzeń (wierzchołek centralny).

\textbf{Definicja} Średnica grafu (długość najdłuższej scieżki w  grafie), jest własnością grafu
$$diam(G) = max\{d(u,v):u,v \in V(G)\}. $$
\textbf{Fakt} W drzewie  średnica grafu jest odległością, pomiędzy dwoma liśćmi, więc w przypadku usunięcia warstwy liści średnica drzewa zmniejszy się o 2.

\textbf{Definicja} Wierzchołek x nazywamy centralnym jeśli jego maksymalna odległość od innych wierzchołków jest najmniejsza, czyli $\max_{y\in V(G)} d(x,y) = \frac{diam(G)+1}{2}$.

\textbf{Lemat} W drzewie T wierzchołkiem centralnym może być pojedyńczy wierzchołek lub dwa sąsiadujące wierzchołki.

\textbf{Dowód} 
Teza: Wierzchołek centralny jest pojedyńczy dla średnicy parzystej, a dla nieparzystej średnicy ma dwa sąsiadujące wierzchołki centralne

Indukcja po N - średnicy drzewa.

Baza dla N = 0 i N = 1 oczywista.

Założenie indukcyjne: Teza jest spełniona dla drzewa o średnicy N.
Teza indukcyjne: Teza jest spełniona dla drzewa o średnicy N+2.

Mamy drzewo T o średnicy N+2, jeśli usuniemy warstwę liści to najdłuższa ścieżka dla każdego wierzchołka zmniejszy się o 1, więc wierzchołki centralne zostaną zachowane, a średnica zmniejszy się do N, a ponieważ parzystość N i N+2 jest taka sama to korzystając z ZI spełniona jest TI.

Na mocy zasady indukcji zachodzi teza.

\section{Algorytm}
Najpierw znajdujemy możliwy korzeń (wierzchołek centralny) w drzewie $T_1$.

Indeksy wszystkich liście drzewa dodajemy na kolejkę $Q_1$, a następnie opróżniamy kolejkę usuwając te liście z drzewa i sąsiednie wierzchołki, które zostały liśćmi dodajemy do kolejki $Q_2$, i powtarzamy ten proces dopóki nie zostanie 1 lub 2 wierzchołki (więcej wierzchołków nie może być bo wtedy istnieje wierzchołek, który nie jest liściem, więc możemy jeszcze usunąć liście). Warto zanotować, że warunek pozostałych wierzchołków sprawdzamy za każdym razem przed usunięciem liści.

Podobnie wykonujemy dla $T_2$.

A następnie jeśli oba drzewa:
\begin{enumerate}
    \item Mają po jednym możliwym korzeniu.
    Wtedy po prostu ukorzeniamy te drzewa w tych wierzchołkach i wykonujemy algorytm podany na wykładzie.
    \item Mają po dwa możliwym korzeniu.
    Wtedy dwa razy uruchamiamy algorytm podany na wykładzie, dla pierwszego korzenia $T_1$ z pierwszym korzeniem $T_2$ oraz dla drugiego korzenia $T_1$ z pierwszym korzeniem $T_2$ (nie trzeba wykonywać sprawdzenia drugiego korzenia $T_1$ z drugim korzeniem $T_2$, ponieważ jest to analogiczne do sprawdzenia obu pierwszych korzeni).
    \item Mają różną liczbę możliwych korzeni.
    Oznacza to, że mają różną liczbę wierzchołków centralnych więc nie mogą być izomorficzne.
\end{enumerate} 
\section{Dowód poprawności}
Wiemy o tym, że jeśli drzewa nie są izomorficzne to obojętnie, gdzie je ukorzenimy algorytm podany na wykładzie zwróci fałsz. Więc będziemy rozważać tylko drzewa izomorficzne $T_1$ i $T_2$ i sprawdzimy, czy dla nich otrzymaliśmy odpowiednie korzenie.

\textbf{Fakt} Z własności izomorfizmu wiemy, że jeśli wierzchołek w $T_1$ ma stopień 1 to jest on izomorficzny z innym wierzchołkiem stopnia 1 w $T_2$. 

Czyli jeśli usuniemy wszystkie liście w $T_1$ i otrzymamy $T'_1$ i podobnie dla $T_2$ otrzymamy $T'_2$ to $T'_1$ i $T'_2$, będą one izomorficzne, ponieważ jeśli mieliśmy mieliśmy bijekcję f określającą, który wierzchołek $T_1$ jest izomorficzny z wierzchołkiem $T_2$ to obcieliśmy naszą funkcję dla wierzchołków z $T'_1$, a f[$T'_1$]= $T'_2$.

Korzystając z tej własności oczywiste jest, że wierzchołek centralny się zachowuje, więc pokażmy, że nasz algorytm znajduje wierzchołek centralny.

Teza: Algorytm znajduje wierzchołek centralny, N = średnica grafu.

Baza indukcji:

Dla N = 0, mamy jeden wierzchołek, więc na pewno jest naszym wierzchołkiem centralnym, co nam zwraca algorytm.
Dla N = 1, mamy dwa wierzchołki, więc oba mogą być naszym wierzchołkiem centralnym, co nam zwraca algorytm. 

Założenie indukcyjne: Dla dowolnego drzewa o średnicy N znajdujemy wierzchołek/i centralny/e.

Teza indukcyjna: Dla dowolnego drzewa o średnicy N+2 możemy znaleźć wierzchołek/i centralny/e.

Weźmy dowolne drzewo o średnicy N+2, jeśli usuniemy warstwe liśći wierzchołek centralny się "zachowa", a my otrzymaliśmy drzewo o średnicy N i korzystając z założenia indukcyjnego znajdujemy wierzchołek centralny.

Więc na mocy zasady o indukcji algorytm znajduje wierzchołek centralny

\section{Złożoność}
Musimy usunąć n-1 lub n-2 wierzchołków, a do tego w przypadku przechowywania informacji tylko w tablicy będziemy musieli przejść tablicę sąsiadów dla każdego usuwanego wierzchołka, czyli każdą krawędź będziemy przechodzili maksymalnie 2 razy, więc nasza złożoność to O(n+m), ale ponieważ w drzewie liczba krawędzi wynosi n-1 to otrzymujemy O(n), dla algorytmu znajdowania wierzchołka centralnego, a sprawdzenie izomorfizmu ukorzenionych drzew też ma złożoność O(n), więc ostatecznie otrzymujemt O(n).

\section{Usprawnienia}
Zamiast tablicy moglibyśmy trzymać w wierzchołkach liste wskaźników na komórki w sąsiedzie, które wskazują na nas, dzięki temu nie musielibyśmy przechodzić wszystkich krawędzi 2 razy, a tylko raz.

Podczas szukania naszego wierzchołka centralnego moglibyśmy równocześnie szukać korzeń w $T_1$ i $T_2$ i przy usuwaniu ,,warstwy'' liści sprawdzać ich liczebność.
\end{document}

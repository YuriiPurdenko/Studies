\documentclass[12pt,a4paper]{article}

\usepackage[T1]{fontenc}
\usepackage[utf8]{inputenc}
\usepackage[polish]{babel}
\usepackage{indentfirst}
\usepackage{amsfonts}
\usepackage{amsmath}
\usepackage{algorithmic}
\usepackage{amssymb}

\usepackage[margin=0.5in,headheight=48pt,top=78pt]{geometry}
\frenchspacing
\setlength{\parskip}{1em}
\setlength{\parindent}{0em}
\def\N{\mathbb{N}}
\def\R{\mathbb{R}}
\newcommand{\zadanie}[1]{\par\textbf{Zadanie #1}}
\newcommand{\odp}[1]{\textbf{Odpowiedź:} #1}

\usepackage{tikz}

\usepackage{fancyhdr}
\pagestyle{fancy}
\fancyhf{} % clear all fields
\fancyhead[C]{ \textbf{AISD} - Lista 7 Zadanie 3\ Wiktor Pilarczyk 308533}

\begin{document}
\section{Wstęp}
Interesuje nas oczekiwana liczba pustych list po umieszczeniu n kluczy w tablicy haszującej $T$ o n elementach. Czyli oznacza to, że w elemencie tablicy może być wiele kluczy. Więc jeśli funkcja haszująca f dla klucza x zwróci wartość $i$ = f(x) to do elementu tablicy T[i], będzie należał $x \in T[i]$, zazwyczaj element tablicy jest listą klucz, które zostały w nim umieszczone. Zakładamy, że nasza funkcja haszująca f z tym samym prawdopodobieństwem dla każdego klucza przydziela elemnt tablicy T.

\section{Obliczanie prawdopodobieństwa}
Chcemy sprawdzić z jakim prawdopodobieństwem element tablicy $T[i]$ jest pusty, niech $P(T[i])$ będzie oznaczało, że prawdopodobieństwem element tablicy $T[i]$ jest pusty, a skoro prawdopodobieństwo, że klucz wyląduje w $T[i]$ wynosi $\frac{1}{n}$ dla każdego klucza. Biorąc zdarzenie przeciwne otrzymujemy wzór:
$$ P(T[i]) = (\frac{n-1}{n})^n$$
\section{Wartość oczekiwana}
Chcemy obliczyć wartość oczekiwaną pustych elementów tablicy T. Najpierw obliczmy wartość oczekiwaną pojedyńczego elemntu:
$$ E(T[i]) = P(T[i]) * 1 + (1 - P(T[i])) * 0 = P(T[i])$$
Mnożymy razy 1, ponieważ jest to przypadek kiedy nasza lista jest pusta, a razy 0 wpp.


Korzystając z liniowości wartości oczekiwanej
$$ E(\sum_{i=0}^n T[i]) = \sum_{i=0}^n E(T[i]) = \sum_{i=0}^n (\frac{n-1}{n})^n = n * (\frac{n-1}{n})^n $$

Wiemy o tym, że dla odpowiednio dużego n:
$$(\frac{n-1}{n})^n -> \frac{1}{e}$$

Więc naszą wartość oczekiwaną możemy oszacować jako:
$$ E(\sum_{i=0}^n T[i]) = \frac{n}{e}$$

\end{document}

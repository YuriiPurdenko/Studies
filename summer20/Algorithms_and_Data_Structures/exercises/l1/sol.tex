\documentclass[12pt,a4paper]{article}

\usepackage[T1]{fontenc}
\usepackage[utf8]{inputenc}
\usepackage[polish]{babel}
\usepackage{indentfirst}
\usepackage{amsfonts}
\usepackage{amsmath}
\usepackage{algorithmic}
\usepackage{amssymb}

\usepackage[margin=0.5in,headheight=48pt,top=78pt]{geometry}
\frenchspacing
\setlength{\parskip}{1em}
\setlength{\parindent}{0em}
\def\N{\mathbb{N}}
\def\R{\mathbb{R}}
\newcommand{\zadanie}[1]{\par\textbf{Zadanie #1}}
\newcommand{\odp}[1]{\textbf{Odpowiedź:} #1}

\usepackage{tikz}

\usepackage{fancyhdr}
\pagestyle{fancy}
\fancyhf{} % clear all fields
\fancyhead[C]{ \textbf{AISD} - Lista 1 Zadanie 5\ Wiktor Pilarczyk 308533}

\begin{document}
\section{Wstęp}
Wiemy, że nasz graf jest acykliczny (czyli jest drzewem/lasem) i jest skierowany. Wiemy o tym, że możemy nasz graf posortować  topologiczne w czasie O(V+E) i oznaczmy wierzchołki $v_1, v_2, ..., v_n$, gdzie dzięki naszemu sortowaniu topologicznemu wiemy, że żadna krawędź nie wychodzi z $v_i$ do $v_j$, gdzie j > i (czyli do $v_n$ nie wchodzi żadna krawędź). 

\section{Algorytm}
Dla każdego wierzchołka mamy tablicę T, która będzie przechowywała wartość najdłuższej drogi wychodzącej z tego wierzchołka. 

Nasz algorytm będzie iterował się po wierzchołkach od $v_1$, do $v_n$ i dla wierzchołka $v_i$, zabierze każdego jego sąsiad $v_j$, takiego że j > i (czyli z j wychodzi krawędź do i) i zaktualizuje wartość w tablicy T:
$$ T[j] = max(T[j], T[i] + e(j,i))$$

Po zakończeniu pętli bierzemy największą wartość z T, dzięki czemu znamy długość najdłuższej drogi oraz indeks oznacza gdzie taka droga może się zaczynac (może być ich wiele).

Przypadek odzyskiwania drogi jest dosyć oczywisty ponieważ skoro znamy długość najdłuższej drogi wychodzącej z danego wierzchołka oraz podobnie dla jego sąsiadów to wiemy, że sąsiad (do którego wychodzi krawędź) dla którego suma krawędzi i jego najdłuższej drogi równa się długości naszej najdłuższej drogi może być elementem tej drogi i tak rekurencyjnie możemy odzyskać tą drogę. Ważne jest warunek, że graf jest acykliczny, ponieważ zapewnia nam to, że otrzymamy drogę.

\section{Dowód}
Teza: Nasz algorytm poprawnie obliczy wartość najdłuższej drogi wychodzącej z każdego wierzchołka.
Przeprowadzę dowód indukcyjny po liczbie wierzchołków.

Baza n = 1
Oczywiste

Założenie indukcyjne: teza jest spełniona dla dowolnego grafu o n wierzchołkach spełniającego założenia

Teza indukcyjna: teza jest spełniona dla dowolnego grafu o n+1 wierzchołkach spełniającego założenia

Weźmy dowolny graf o n+1 wierzchołkach posortujmy go topologicznie $v_1, v_2, ..., v_{n+1}$, weźmy podgraf bez $v_{n+1}$ jest on n elementowy i korzystając z ZI dla każdego wierzchołka $v_1, v_2, ..., v_n$ teza jest spełniona, ponieważ z wierzchołka $v_{n+1}$ tylko wychodzą krawędzie. Załóżmy nie wprost, że nasza wartość obliczona dla $v_{n+1}$ nie jest optymalna. Weźmy tą optymalną i należy do niej sąsiad $v_{n+1}$ - $v_i$, ale $T[i] + d(n+1, i)$ rozważaliśmy, więc otrzymujemy sprzeczność. Czyli dobrze obliczyliśmy najdłuższą drogę dla $v_{n+1}$.

Na mocy zasady o indukcji teza zachodzi dla dowolnego grafu.

\section{Złożoność}
Wiemy już, że sortowanie topologiczne ma złożoność O(V+E). Obliczanie wartości najdłuższej drogi wychodzącej z wierzchołka kosztuje nas O(V+E), ponieważ odwiedzamy każdy wierzchołek iteracyjnie, a dla wierzchołka sprawdzamy krawędzie, które będą przechodzone dwukrotnie (dla wierzchołków połączonych tą krawędzią) i podobnie dla wypisywania tej ścieżki. Więc złożoność wynosi O(V+E).
\end{document}

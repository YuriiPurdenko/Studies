\documentclass[a4paper]{article}
% Kodowanie latain 2
%\usepackage[latin2]{inputenc}
\usepackage[T1]{fontenc}
% Można też użyć UTF-8
\usepackage[utf8]{inputenc}
\usepackage{filecontents,listings,graphicx,varwidth}% http://ctan.org/pkg/{filecontents,listings,graphicx,varwidth}
% Język
\usepackage[polish]{babel}
% \usepackage[english]{babel}

% Rózne przydatne paczki:
% - znaczki matematyczne
\usepackage{amsmath, amsfonts}
% - wcięcie na początku pierwszego akapitu
\usepackage{indentfirst}
% - komenda \url 
\usepackage{hyperref}
% - dołączanie obrazków
\usepackage{graphics}
% - szersza strona
\usepackage[nofoot,hdivide={2cm,*,2cm},vdivide={2cm,*,2cm}]{geometry}
\frenchspacing
% - brak numerów stron
\pagestyle{empty}

% dane autora
\author{Wiktor Pilarczyk, UT: 923434620}
\title{Przykładowy plik w systemie \LaTeX}
\date{\today}

% początek dokumentu
\begin{document}
\maketitle
\section{Polecenie ID}
Wyświelta rzeczywiste ID użytkownika, aktualnie zalogowanego (UID), unikalny numer przypisany do danego użytkownika w systemie) oraz  grupy do niego przypisanej(GID). Reszta to są grupy do, których użytkownik należy wraz z ich numerami.
\section{Czas Unixowy}
Jest to system używany do opisywania daty. Polega na liczbie sekund, która upłyneła od 1 stycznia 1970 roku. Problem roku 2038 polega na tym, że w systemach gdzie ta liczba jest trzymana przez 32 bity, dnia 19 stycznia 2038 roku o godzinie 3:14:08, liczba przekroczy zakres 2147483647 ($2^{31}-1$).
\section{Wielokrotna Kompilacja}
Pierwsze kompilacja aktualizuje referencje, a druga umieszcza referencje.
\section{ASCII ART}
Obrazek:\\
:~~~~\slash\textbackslash\_\slash\textbackslash ~~~ (\\
:~~~(~\^ ~.~ \^~)~\_)\\
:~~~~~~\textbackslash " \slash ~~~~(\\
:~~~(~~|~~~|~~~)\\
:~(\_\_d~b\_\_)\\
\end{document}
